\section{Related Work}
\label{sec:related_work}

Although web-based visualizers can be found for many different domains, there are only a few that have similar capabilities to those we describe here--namely, sharing animations with other collaborators and other researchers.
%
\url{Clara.io}~\autocite{Clara.2018.SIGGRAPH.Web} is a cloud-based web-application for modeling, animating, and rendering scenes.
%
Clara.io has an impressive list of features, including the ability to create models and scenes in the browser. With Clara.io you can send animation links to collaborators, but the data files must be hosted by Clara.io and the project is not open source.
%
More importantly, Clara.io uses standard graphics formats, which makes it difficult to generate animation data as part of an evolutionary experiment work-flow.
%
Sketchfab~\autocite{Sketchfab.2018.Web} is another alternative, and like Clara.io, Sketchfab hosts the data files and works with common graphics files.
%
While both of would enable the sharing of visualizations, both would require an extensive amount of work to generate visualization data.
%
Another drawback of these websites is that they require a paid account to keep any files private, whereas with Review a researcher can choose to only share the log files with specific collaborators.
