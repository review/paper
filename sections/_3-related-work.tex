\section{Related Work}
\label{sec:related_work}

Although web-based visualizers can be found for many different domains, there are only a few that have similar capabilities to those we describe here--namely, sharing animations with other collaborators and other researchers.
%
\url{Clara.io}~\autocite{Clara.2018.SIGGRAPH.Web} is a cloud-based web-application for modeling, animating, and rendering scenes.
%
Clara.io has an impressive list of features, including the ability to create models and scenes in the browser. With Clara.io you can send animation links to collaborators, but the data files must be hosted by Clara.io and the project is not open source.
%
More importantly, Clara.io uses standard graphics formats, which makes it difficult to generate animation data as part of an evolutionary experiment work-flow.
%
Sketchfab~\autocite{Sketchfab.2018.Web} is another alternative, and like Clara.io, Sketchfab hosts the data files and works with common graphics files.
%
In contrast to Clara.io, Sketchfab focuses more on sharing and hosting graphics demos and less on authoring new assets and models.
%
Verge3D~\autocite{Kovelenov.2018.Verge3D} is an upcoming product that has similar features to both Clara.io and Sketchfab. Verge3D, however, focuses more on providing a means for creating 3D Web applications hosted by users. For example, it can be used to create an e-commerce website that delivers interactive 3D renderings of products.


Apart from these larger projects, Shen's Clay-Viewer~\autocite{Shen.2018.ClayViewer} and McCurdy's glTF Viewer~\autocite{McCurdy.2017.glTFViewer} have a similar interface to Review, and they load local animation files in glTF format.
%
None of the above applications, however, provide a feature similar to our HTTP URI query component whereby an animation can be loaded by providing a link to the animation data in the URI.
%
While these alternatives would enable sharing visualizations, they all require an extensive amount of work to generate visualization data (see the discussion on glTF in section~\ref{sec:review}).
%
Another drawback of these websites is that many require a paid account to keep hosted files private, whereas with Review a researcher can choose to only share the log files with specific collaborators.
