% Maximum 200 words

\begin{abstract}

%%% Background
% This section should be the shortest part of the abstract and should very briefly outline the following information:
% - What is already known about the subject, related to the paper in question
% - What is not known about the subject and hence what the study intended to examine (or what the paper seeks to present)
Evolutionary robotics researchers often need to share results that may be too difficult to describe in text and too complex to show using images. Many researchers include links to videos as supplementary materials, but videos have a predefined view of the scene and do not allow watchers to adjust the viewing angle to their preference.
%
%
%
%%% Methods
% The methods section is usually the second-longest section in the abstract. It should contain enough information to enable the reader to understand what was done, and how.
In this paper we present a web-based application (based on three.js) for sharing interactive animations. Specifically, our tool (called Review) enables researchers to generate simple animation log data that can be loaded in any modern web browser and computer or mobile phone. The camera in these animations can be controlled by the user such that they can pan, tilt, rotate, and zoom in and out of the scene.
%
%
%
% Results
% The results section should be the longest part of the abstract and should contain as much detail about the findings as the journal word count permits.
%
%
%
% Conclusion
% This section should contain the most important take-home message of the study, expressed in a few precisely worded sentences. Usually, the finding highlighted here relates to the primary outcome measure; however, other important or unexpected findings should also be mentioned. It is also customary, but not essential, for the authors to express an opinion about the theoretical or practical implications of the findings, or the importance of their findings for the field. Thus, the conclusions may contain three elements:
% - The primary take-home message
% - The additional findings of importance
% - The perspective
Review is meant to improve the ability of researchers to share their evolved results with one another.

\end{abstract}
